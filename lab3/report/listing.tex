В листинге \ref{lst:prog} представлен фрагмент кода программы, отвечающий за генерирование последовательностей и вычисление критерия.

\begin{lstlisting}[caption = {Вычисление функций распредлений}, label=lst:prog]
class RndAlgGen: IRndGen
{
	private int seed;
	public RndAlgGen(int minN, int maxN, int s=0)
	{
		MinN = minN;
		MaxN = maxN;
		SetSid(s);
	}
	
	public override void SetSid(int s)
	{
		seed = s;
	}
	
	public override int Rand()
	{
		seed = seed * 16807 + 2147483647;
		var res = ((uint)seed >> 16) % (MaxN - MinN) + MinN;
		return (int)res;
	}
	
	public override int[] RandArr(int size)
	{
		int[] res = new int[size];
		for (int i = 0; i < size; i++)
		res[i] = Rand();
		return res;
	}
}

class RndTabGen : IRndGen
{
	private int seed;
	private StreamReader f;
	private string fpath;
	
	public static void FillFile(string fpath_, int min, int max)
	{
		var rand = new Random();
		using (StreamWriter sw = File.CreateText(fpath_))
		{
			for (int i = 0; i < (max - min) * 1000; i++)
			sw.WriteLine(rand.Next(min, max));
		}
	}
	public RndTabGen(int minN, int maxN, string fp, int s = 0)
	{
		MinN = minN;
		MaxN = maxN;
		fpath = fp;
		SetSid(s);
		var chi = new ChiSquared(1.0);
	}
	
	private void OpenFile(string fp)
	{
		fpath = fp;
		if (f != null)
		f.Close();
		f = File.OpenText(fpath);
		for (int i = 0; i < seed; i++)
		f.ReadLine();
	}
	
	public override void SetSid(int s)
	{
		seed = s;
		OpenFile(fpath);
	}
	
	public override int Rand()
	{
		string fileStr = f.ReadLine();
		if (fileStr == null)
		{
			SetSid(0);
			fileStr = f.ReadLine();
		}
		seed++;
		
		var res = (UInt32.Parse(fileStr)) % (MaxN - MinN) + MinN;
		return (int)res;
	}
	
	public override int[] RandArr(int size)
	{
		int[] res = new int[size];
		for (int i = 0; i < size; i++)
		res[i] = Rand();
		return res;
	}
}


class Crit
{
	public int[] arr;
	public Crit(int[] arr_)
	{
		arr = arr_;
	}
	
	public double Value(int min, int max)
	{
		int n = arr.Length;
		double p = 1.0 / (max - min);
		double acc = 0;
		for (int i = min; i < max; i++)
		acc += Math.Pow(arr.Count(x => x == i), 2) / p;
		acc = acc / (double)n - n;
		
		double res = ChiSquared.CDF(max - min - 1, acc);
		return res;
	}
}
\end{lstlisting}
