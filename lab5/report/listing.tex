В листинге \ref{lst:prog} представлен фрагмент кода программы, отвечающий за моделирование.

\begin{lstlisting}[caption = {Реализация модели}, label=lst:prog]
public class EGenerated : Event
{
	public EGenerated(double t_) : base(EventTypes.NewClient, t_) {}
	public override void Handle(EventModel model)
	{
		Request req = model.Gen.New(model.CurT);
		model.createdN++;
		if (model.createdN < model.maxCreatedN)
		model.AddEvent(new EGenerated(model.Gen.WhenReady()));
		
		for (int i = 0; i < model.Operators.Count; i++)
		{
			var oper = model.Operators[i];
			if (oper.IsFree())
			{
				oper.Put(req, model.CurT);
				model.AddEvent(new EOperatorServed(oper.WhenReady(), i));
				return;
			}
		}
		
		model.denyedN++;
	}
}

public class EOperatorServed: Event
{
	private int num;
	public EOperatorServed(double t_, int num_) : base(EventTypes.OperatorServed, t_) 
	{
		this.num = num_;
	}
	
	public override void Handle(EventModel model)
	{
		Request req = model.Operators[num].Get();
		int targetQueue = (num == 2) ? 1 : 0;
		model.CompQueue[targetQueue].Push(req, model.CurT);
		if (model.Computers[targetQueue].IsFree())
		model.AddEvent(new EComputerServed(model.CurT, targetQueue));
	}
}

public class EComputerServed : Event
{
	private int num;
	public EComputerServed(double t_, int num_) : base(EventTypes.ComputerServed, t_) 
	{
		this.num = num_;
	}
	public override void Handle(EventModel model)
	{
		Request req = model.Computers[num].Get();
		if (req != null)
		model.servedN++;
		
		if (model.CompQueue[num].Count == 0)
		return;
		req = model.CompQueue[num].Pop();
		if (req != null)
		{
			model.Computers[num].Put(req, model.CurT);
			model.AddEvent(new EComputerServed(model.Computers[num].WhenReady(), num));
		}
	}
}

public class EventModel
{
	public double CurT;
	public Generator Gen;
	public List<Service> Operators;
	public List<Service> Computers;
	public List<ReqQueue> CompQueue;
	
	public int createdN = 0;
	public int servedN = 0;
	public int denyedN = 0;
	public int maxCreatedN = 300;
	
	private List<Event> Events;
	
	public EventModel(Generator generator, List<Service> operators, List<Service> computers)
	{
		Gen = generator;
		Operators = operators;
		Computers = computers;
		CompQueue = new List<ReqQueue> { new ReqQueue(), new ReqQueue() };
	}
	
	private void Reset()
	{
		createdN = 0;
		servedN = 0;
		denyedN = 0;
		CurT = 0;
		
		Gen.New(CurT);
		Events = new List<Event> { new EGenerated(Gen.WhenReady()) };
	}
	
	public ModelingResult Run()
	{
		Reset();
		while (Events.Count > 0)
		{
			Event e = this.Events[0];
			this.Events.RemoveAt(0);
			
			CurT = e.time;
			e.Handle(this);
		}
		
		return new ModelingResult(this.servedN, this.denyedN, this.CurT);
	}
	
	public void AddEvent(Event e)
	{
		Events.Add(e);
		Events.Sort();
	}
}
\end{lstlisting}
