В листинге \ref{lst:prog} представлен фрагмент кода программы, отвечающий за моделирование цепей Маркова.

\begin{lstlisting}[caption = {Вычисление функций распредлений}, label=lst:prog]
class Model
{
	public const int maxN = 10;
	public double[] pArr;
	public double[] tArr;
	public double[,] tMatrix;
	public int n { get; }
	public double T = 0;
	
	public Model(int n_)
	{
		this.n = n_;
		this.tMatrix = new double[maxN, maxN];
		this.tArr = new double[maxN];
		this.pArr = new double[maxN];
		pArr[0] = 1;
	}
	
	// returns is stabilized
	public bool Step(double dt)
	{
		double[] newP = new double[maxN];
		for (int i=0; i < n; i++)
		{
			newP[i] = pArr[i];
			for (int j = 0; j < n; j++)
			{
				if (i == j) continue;
				newP[i] += dt * (pArr[j] * tMatrix[j, i] - pArr[i] * tMatrix[i, j]);
			}
		}
				
		pArr = newP;
		T += dt;
		
		SetStableT();
		return IsStable();
	}
	
	private bool IsStable()
	{
		double[] res = Kolmogorov();
			for (int i = 0; i < n; i++)
				if (Math.Abs(res[i]) > 1e-8)
					return false;
		return true;
	}
	
	public double[] Kolmogorov()
	{
		double[] res = new double[maxN];
		
		for (int i = 0; i < n; i++)
		{
			res[i] = 0;
			for (int j = 0; j < n; j++)
			res[i] += pArr[j] * tMatrix[j, i] - pArr[i] * tMatrix[i, j];
		}
		
		return res;
	}
	
	private void SetStableT()
	{
		double[] kArr = Kolmogorov();
		for (int i = 0; i < n; i++)
		{
			if (Math.Abs(kArr[i]) < 1e-5 && tArr[i] <= 1e-7)
			tArr[i] = T;
			else if (Math.Abs(kArr[i]) > 1e-5 && tArr[i] > 1e-7)
			tArr[i] = 0;
		}
	}
}
\end{lstlisting}
